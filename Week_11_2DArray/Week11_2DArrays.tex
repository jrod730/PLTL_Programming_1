\documentclass[12pt]{article}
\usepackage[margin=0.8in]{geometry}
\usepackage{amsmath}
\usepackage{enumerate}
\usepackage{enumitem}
\usepackage[pdftex]{graphicx}
\usepackage{hyperref}
\usepackage{framed}
\usepackage{upquote}
\usepackage{listings}
\usepackage{makecell}
\usepackage{pxfonts}
\usepackage[T1]{fontenc}
\usepackage{booktabs}
\pagestyle{plain}

\lstset{%
  language=Java,
  basicstyle=\fontsize{9}{11}\selectfont\ttfamily,
  keywordstyle=\bfseries,
  breaklines=true,
  commentstyle = \fontsize{9}{11}\selectfont\ttfamily,
  columns=fullflexible,
  frame=single,
  showstringspaces=false,
  tabsize=4,
  morekeywords = {main},
}

\begin{document}
\begin{center}
	\textbf{CS-200: Programming I}\\
	\textbf{Spring 2017}\\
	\textbf{Northeastern Illinois University}\\
	\textbf{PLTL: Week of 04/03/17}\\
	\textbf{2D Arrays}
\end{center}


\noindent\underline{\textbf{Problem \#1}}
\begin{itemize}
	\item Write a program that has the class name \texttt{Problem1} and that has the \texttt{main} method. Leave the \texttt{main} method empty for now.
	\item Write a method named \texttt{binaryArr} that takes one parameter, an 2-dimensional (2D) boolean array named \texttt{a} and returns a 2D integer array.
	\item The method should create a new integer array that replaces every \texttt{false} from the booelan array with a \texttt{0} and every \texttt{true} with a \texttt{1}.
	\item Create a \texttt{print2DArray} method that takes a 2D integer array as a parameter and prints out the elements of each row on its own line separated by spaces.
	\item Several sample usages are provided for you below. Use the sample usages in the \texttt{main} method to test your code (and use the \texttt{printArray} method to print out the results of calling the \texttt{transpose} method!).
\end{itemize}
\begin{table}[htbp]
  \centering
    \begin{tabular}{|p{28.055em}|p{5.13em}|}
    \toprule
    \multicolumn{1}{|c|}{Sample Method Usage} & \multicolumn{1}{l|}{Output} \\
    \midrule
    \texttt{\textbf{boolean}[][] b1 = \{\{\textbf{false, true, false, true, true}\}, \newline \hspace*{3.75cm} \{\textbf{true, false, false, true, true}\},\newline \hspace*{3.75cm} \{\textbf{false, false, true, true, false}\}\};}  & \texttt{01011\newline{}10011\newline{}00110} \\
    \midrule
    \texttt{\textbf{boolean}[][] b1 =\{\{\textbf{true, false, true, false}\}, \newline \hspace*{3.55cm} \{\textbf{false, false, true, false}\}\};} & \texttt{1010\newline{}0010} \\
    \bottomrule
    \end{tabular}%
  \label{tab:addlabel}%
\end{table}%

\vspace*{0.5cm}
\noindent\underline{\textbf{Problem \#2}}
\begin{itemize}
	\item Write a program that has the class name \texttt{Problem2} and that has the \texttt{main} method. Leave the \texttt{main} method empty for now.
	\item Write a method named \texttt{inSequence} that takes a 2D integer array, and returns a boolean value. 
	\item The method should check if it the numbers are in sequence from  1 to n * n . If they are, then return true. You may assume that the array is already a perfect square.
	\item Several sample usages are provided for you below. Use the sample usages in the \texttt{main} method to test your code.
\end{itemize}

\begin{table}[htbp]
  \centering
    \begin{tabular}{|p{20.66em}|c|}
    \toprule
    \multicolumn{1}{|c|}{Sample Method Usage} & Return Value \\
    \midrule
    \texttt{\textbf{int} [][] a1 = \{\{1, 2, 3,  4\},\newline \hspace*{3.30cm}\{5, 6, 7, 8\},\newline \hspace*{3.30cm}\{9, 10, 11, 12\}, \newline \hspace*{3.30cm}\{13, 14, 15, 16\}\}; }& \texttt{\textbf{true}} \\
    \midrule
    \texttt{\textbf{int} [][] a1 = \{\{1, 3, 2,  4\},\newline \hspace*{3.1cm} \{3, 5, 8, 7\},\newline \hspace*{3.1cm} \{12, 11, 16, 8\},\newline \hspace*{3.1cm} \{12, 14, 15, 16\}\};}& \texttt{\textbf{false}} \\
    \bottomrule
    \end{tabular}%
  \label{tab:addlabel}%
\end{table}%

\end{document}