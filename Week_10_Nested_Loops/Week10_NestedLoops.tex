\documentclass[12pt]{article}
\usepackage[margin=0.8in]{geometry}
\usepackage{amsmath}
\usepackage{enumerate}
\usepackage{enumitem}
\usepackage[pdftex]{graphicx}
\usepackage{hyperref}
\usepackage{framed}
\usepackage{upquote}
\usepackage{listings}
\usepackage{makecell}
\usepackage{pxfonts}
\usepackage[T1]{fontenc}
\pagestyle{plain}

\lstset{%
  language=Java,
  basicstyle=\fontsize{9}{11}\selectfont\ttfamily,
  keywordstyle=\bfseries,
  breaklines=true,
  commentstyle = \fontsize{9}{11}\selectfont\ttfamily,
  columns=fullflexible,
  frame=single,
  showstringspaces=false,
  tabsize=4,
  morekeywords = {main},
}

\begin{document}
\begin{center}
	\textbf{CS-200: Programming I}\\
	\textbf{Spring 2017}\\
	\textbf{Northeastern Illinois University}\\
	\textbf{PLTL: Week of 03/05/17}\\
	\textbf{Arrays/Methods}
\end{center}


\noindent\underline{\textbf{Problem \#1}}
\begin{itemize}
	\item Write a program that has the class name \texttt{Problem1} and that has the \texttt{main} method. Prompt the user to enter a integer between 1 and 4 with both, row and column.
	\item The program should create a multiplication table by finding the product for each incrementation of the row and column respectively. The table should end at \texttt{row * column}.
	\item Several sample usages are provided for you below. 
\end{itemize}
\begin{center}
\begin{minipage}{3cm}
\begin{lstlisting}[escapechar=@]
Enter row: @\textbf{3}@
Enter col: @\textbf{4}@
1 2 3 4 
2 4 6 8 
3 6 9 12 
\end{lstlisting}
\end{minipage}
\hspace*{.5cm}
\begin{minipage}{3cm}
\begin{lstlisting}[escapechar=@]
Enter row: @\textbf{3}@
Enter col: @\textbf{3}@
1 2 3 
2 4 6 
3 6 9
\end{lstlisting}
\end{minipage}
\hspace*{.5cm}
\begin{minipage}{3cm}
\begin{lstlisting}[escapechar=@]
Enter row: @\textbf{4}@
Enter col: @\textbf{4}@
1 2 3 4 
2 4 6 8 
3 6 9 12 
4 8 12 16 
\end{lstlisting}

\end{minipage}\\


\end{center}

\vspace*{0.5cm}
\noindent\underline{\textbf{Problem \#2}}
\begin{itemize}
	\item Write a program that has the class name \texttt{Problem2} and that has the \texttt{main} method. 
	\item Write a program that creates a 10 by 10 box. The box should look exactly as the sample usage below. 
	\item Even though the values are given nothing should be hard coded all constants must be assigned a variable.
	\item Several sample usages are provided for you below. 
\end{itemize}
\begin{center}
\hspace*{.5cm}
\begin{minipage}{10cm}
\begin{lstlisting}[escapechar=@]
1@\hspace*{.34cm}@2@\hspace*{.33cm}@3@\hspace*{.33cm}@4@\hspace*{.34cm}@5@\hspace*{.34cm}@6@\hspace*{.34cm}@7@\hspace*{.34cm}@8@\hspace*{.3cm}@9@\hspace*{.34cm}@10
2   @\hspace*{4cm}@   20 
3   @\hspace*{4cm}@   30 
4   @\hspace*{4cm}@   40 
5   @\hspace*{4cm}@   50 
6   @\hspace*{4cm}@   60 
7   @\hspace*{4cm}@   70 
8   @\hspace*{4cm}@   80 
9   @\hspace*{4cm}@   90 
10 20 30 40 50 60 70 80 90 100 
\end{lstlisting}
\end{minipage}\\
\end{center}

\end{document}