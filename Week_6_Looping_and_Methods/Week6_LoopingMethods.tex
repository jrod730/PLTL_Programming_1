\documentclass[12pt]{article}
\usepackage[margin=0.8in]{geometry}
\usepackage{amsmath}
\usepackage{enumerate}
\usepackage{enumitem}
\usepackage[pdftex]{graphicx}
\usepackage{hyperref}
\usepackage{framed}
\usepackage{listings}
\usepackage{makecell}
\usepackage{pxfonts}
\pagestyle{plain}

\lstset{%
  language=Java,
  basicstyle=\fontsize{9}{11}\selectfont\ttfamily,
  keywordstyle=\bfseries,
  breaklines=true,
  commentstyle = \fontsize{9}{11}\selectfont\ttfamily,
  columns=fullflexible,
  frame=single,
  showstringspaces=false,
  tabsize=4,
  morekeywords = {main},
}

\begin{document}
\begin{center}
	\textbf{CS-200: Programming I}\\
	\textbf{Fall 2017}\\
	\textbf{Northeastern Illinois University}\\
	\textbf{PLTL: Week of 10/03/17}\\
	\textbf{Looping/Methods}
\end{center}


\noindent\textbf{Problem \#1}
\begin{itemize}
	\item Write a program that has the class name \texttt{Problem1} and that has the \texttt{main} method.
	\item Prompt the user to enter \texttt{int n} less than 20
	\item Create new method named \texttt{powerBaseTwo} which takes \texttt{int} as a parameter and returns an \texttt{int}. The program should also print the result. You may not use \texttt{Math.pow} method and you can only use decrementation operators.
	\item Create a third method named \texttt{isDivByFive} which takes \texttt{int} as parameter and returns void. This method should check if the answer printed in \texttt{powerBaseTwo} method is divisible by 4. If it is, then print the quotient. If is it not, then print \texttt{-1}.
	\item Several sample runs are provided for you below. Your output must be formatted \textbf{exactly} like the sample runs below. 
\end{itemize}
\begin{center}
\begin{minipage}{4cm}
\begin{lstlisting}[escapechar=@]
Enter int (<20): @\textbf{9}@
512
128
\end{lstlisting}
\end{minipage}
\hspace*{0.5cm}
\begin{minipage}{4cm}
\begin{lstlisting}[escapechar=@]
Enter int (<20): @\textbf{8}@
256
64
\end{lstlisting}
\end{minipage}
\hspace*{0.5cm}
\begin{minipage}{4cm}
\begin{lstlisting}[escapechar=@]
Enter int (<20): @\textbf{0}@
1
-1
\end{lstlisting}
\end{minipage}\\

\end{center}

\vspace*{0.5cm}
\noindent\textbf{Problem \#2}
\begin{itemize}
	\item Write a program that has the class name \texttt{Problem2} and that has the \texttt{main} method. 
	\item Prompt the user to enter 5 integers.
	\item Create a second method named \texttt{sumEveryDigit} which takes an \texttt{int} as a parameter and returns an \texttt{int}. This method should sum every digit in all of the integers enter by the user then return the sum.
	\item The total sum from every digit entered should be printed.
	\item Several sample usages are provided for you below. Use the sample usages in the \texttt{main} method to test your code.
\end{itemize}
\begin{center}
\begin{minipage}{7cm}
\begin{lstlisting}[escapechar=@]
Enter 5 Integers: @\textbf{ 1 2 3 4 5}@
15
\end{lstlisting}
\end{minipage}
\hspace*{0.5cm}
\begin{minipage}{7cm}
\begin{lstlisting}[escapechar=@]
Enter 5 Integers: @\textbf{ 0 0 0 0 1234}@
10
\end{lstlisting}
\end{minipage}
\end{center}

\end{document}