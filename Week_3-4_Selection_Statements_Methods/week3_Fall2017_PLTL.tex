\documentclass[12pt]{article}
\usepackage[margin=0.7in]{geometry}
\usepackage{amsmath}
\usepackage{enumerate}
\usepackage{enumitem}
\usepackage[pdftex]{graphicx}
\usepackage{hyperref}
\usepackage{framed}
\usepackage{listings}
\usepackage{makecell}
\usepackage{pxfonts}
\pagestyle{plain}

\lstset{%
  language=Java,
  basicstyle=\fontsize{9}{11}\selectfont\ttfamily,
  keywordstyle=\bfseries,
  breaklines=true,
  commentstyle = \fontsize{9}{11}\selectfont\ttfamily,
  columns=fullflexible,
  frame=single,
  showstringspaces=false,
  tabsize=4,
  morekeywords = {main},
}

\begin{document}
\begin{center}
	\textbf{CS-200: Programming I}\\
	\textbf{Fall 2017}\\
	\textbf{Northeastern Illinois University}\\
	\textbf{PLTL: Week of 11/21/17}\\
	\textbf{Selection Statements/Methods}
\end{center}

\noindent\textbf{Practice Problem \#1}
\begin{itemize}
	\item Write a program that has the class name \texttt{Problem1} and that has the \texttt{main} method.
	\item In the main method prompt the user for\texttt{ n1} and \texttt{n2}.
	\item Write a second method named \texttt{max} that takes two integer parameters, \texttt{a} and \texttt{b}.
	\item The method should return whichever integer is larger wihtout using the Math class. The method should also display which integer was larger or that they are equal.
	\item Several sample method calls are provided for you below. You should test your method inside the \texttt{main} method.
\end{itemize}
\begin{center}
\begin{tabular}{| c | c |}
\hline\rule{0pt}{4ex}
Sample Method Usage & Output \\
\hline\rule{0pt}{4ex}
\texttt{max(5, 5)} & \texttt{They are equal}\\
\hline\rule{0pt}{4ex}
\texttt{max(-4, 3)} & \texttt{n2 is the max at: 3}\\
\hline\rule{0pt}{4ex}
\texttt{max(6, 11)} & \texttt{n2 is the max at: 11}\\
\hline\rule{0pt}{5ex}
\texttt{max(8, 2)} & \texttt{n1 is the max at: 8}\\
\hline
\end{tabular}
\end{center}

\vspace*{0.5cm}
\noindent\textbf{Practice Problem \#2}
\begin{itemize}
	\item Write a program that has the class name \texttt{Problem2} and that has the \texttt{main} method. This method should take user input. You may assume the integer that is input is less than 100.
	\item Write a second method named \texttt{sumOfDigits} that takes one integer parameter, \texttt{a}.
	\item The method should return the sum of the digits entered by the user.
	\item Several sample method calls are provided for you below. You should test your method inside the \texttt{main} method.
\end{itemize}
\begin{center}
\begin{tabular}{| c | c |}
\hline\rule{0pt}{4ex}
Sample Method Usage & Output \\
\hline\rule{0pt}{4ex}
\texttt{sumOfDigits(10)} & \texttt{Sum is: 1}\\
\hline\rule{0pt}{4ex}
\texttt{sumOfDigits(99)} & \texttt{Sum is: 18}\\
\hline\rule{0pt}{4ex}
\texttt{sumOfDigits(25)} & \texttt{Sum is: 7}\\
\hline
\end{tabular}
\end{center}

\end{document}