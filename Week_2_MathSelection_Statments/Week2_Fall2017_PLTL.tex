\documentclass[12pt]{article}
\usepackage[margin=0.7in]{geometry}
\usepackage{amsmath}
\usepackage{enumerate}
\usepackage{enumitem}
\usepackage[pdftex]{graphicx}
\usepackage{hyperref}
\usepackage{framed}
\usepackage{listings}
\usepackage{pxfonts}
\pagestyle{plain}

\lstset{%
  language=Java,
  basicstyle=\fontsize{9}{11}\selectfont\ttfamily,
  keywordstyle=\bfseries,
  breaklines=true,
  commentstyle = \fontsize{9}{11}\selectfont\ttfamily,
  columns=fullflexible,
  frame=single,
  showstringspaces=false,
  tabsize=4,
  morekeywords = {main},
}

\begin{document}
\begin{center}
	\textbf{CS-200: Programming I}\\
	\textbf{Fall 2017}\\
	\textbf{Northeastern Illinois University}\\
	\textbf{PLTL: Week of 11/14/17}\\
	\textbf{Selection Statements/Math}
\end{center}

\noindent\textbf{Practice Problem \#1}
\begin{itemize}
	\item Write a program that has the class name \texttt{CheckIfSame} and that has the \texttt{main} method.
	\item The program should ask the user to enter two integers.
	\item The program should determine whether the two integers have have the same quotient when divided by 10 and check if the same two integers are divisible by ten.
	\item If they have the same quotient print out \texttt{The quotients are the same!}.
	\item If they are both divisible by 10, print out \texttt{They are both divisible by ten!}.
	\item The pogram should then check if the two integers are either divisble by 10 or have the same quotient when divided by 10. If they do, print out \texttt{Either quotient or remainder are the same!}. 
	\item If the two integers are both divisible by 10 and have the same quotient, then check if both quotients are divisible by 2. If they are print \texttt{The quotient is divisible by two!}. If they are not, print \texttt{The quotient is not divisible by 2.}.
	\item If they don't have the same quotient and are not divisible by 10 then print \texttt{Nothing is the same!}.
	\item Several sample runs are provided for you below. Format your output to match the sample output.
\end{itemize}
\begin{center}
\begin{minipage}{6cm}
\begin{lstlisting}[escapechar=@]
First number: @\textbf{10}@
Second number: @\textbf{10}@
The quotients are the same!
They are both divisible by ten!
The quotient is not divisible by 2.
\end{lstlisting}
\end{minipage}
\hspace*{0.5cm}
\begin{minipage}{7cm}
\begin{lstlisting}[escapechar=@]
First number: @\textbf{3}@
Second number: @\textbf{5}@
The quotients are the same!
Either quotient or remainder are the same!
\end{lstlisting}
\end{minipage}
\end{center}

\vspace*{0.5cm}
\noindent\textbf{Practice Problem \#2}
\begin{itemize}
	\item Write a program that has the class name \texttt{MaxMinMax} and that has the \texttt{main} method.
	\item The program should take the absolute value of the integers.
	\item The program should check which absolute value is larger and then display that value accordingly (see output). If they are equal the program should print that they are equal.
	\item Use the following equation to print a new maximum value (see output): $Max^{Min}$, $Min^{Max}$
	\item Several sample runs are provided for you below. Format your output to match the sample output.
\end{itemize}
\begin{center}
\begin{minipage}{4cm}
\begin{lstlisting}[escapechar=!]
Enter n1: !\textbf{2.0}!
Enter n2: !\textbf{2.0}!
Their values are equal.
\end{lstlisting}
\end{minipage}
\hspace*{0.5cm}
\begin{minipage}{7cm}
\begin{lstlisting}[escapechar=!]
Enter n1: !\textbf{2.0}!
Enter n2: !\textbf{4.0}!
Absolute value of 4.0 is greater than 2.0
The new max is: 16.0
\end{lstlisting}
\end{minipage} \\
\vspace*{0.1cm}
\begin{minipage}{7cm}
\begin{lstlisting}[escapechar=!]
Enter n1: !\textbf{10.0}!
Enter n2: !\textbf{5.0}!
Absolute value of 10.0 is greater than 5.0
The new Max is: 9765625.0
\end{lstlisting}
\end{minipage} 
\end{center}	

\end{document}