\documentclass[12pt]{article}
\usepackage[margin=0.8in]{geometry}
\usepackage{amsmath}
\usepackage{enumerate}
\usepackage{enumitem}
\usepackage[pdftex]{graphicx}
\usepackage{hyperref}
\usepackage{framed}
\usepackage{upquote}
\usepackage{listings}
\usepackage{makecell}
\usepackage{pxfonts}
\usepackage[T1]{fontenc}
\pagestyle{plain}

\lstset{%
  language=Java,
  basicstyle=\fontsize{9}{11}\selectfont\ttfamily,
  keywordstyle=\bfseries,
  breaklines=true,
  commentstyle = \fontsize{9}{11}\selectfont\ttfamily,
  columns=fullflexible,
  frame=single,
  showstringspaces=false,
  tabsize=4,
  morekeywords = {main},
}

\begin{document}
\begin{center}
	\textbf{CS-200: Programming I}\\
	\textbf{Spring 2017}\\
	\textbf{Northeastern Illinois University}\\
	\textbf{PLTL: Week of 03/05/17}\\
	\textbf{Arrays/Methods}
\end{center}


\noindent\underline{\textbf{Problem \#1}}
\begin{itemize}
	\item Write a program that has the class name \texttt{Problem1} and that has the \texttt{main} method. Leave the \texttt{main} method empty for now.
	\item Write a method named \texttt{centeredAverage} that takes one parameter, an integer array \texttt{a} and returns an integer.
	\item The method should find the average of the elements in the array by omitting the largest and smallest integers . If there are multiple smallest and largest numbers you may ignore only one of each respectively.
	\item Return the average using integer division.
	\item Several sample usages are provided for you below. Use the sample usages in the \texttt{main} method to test your code.
\end{itemize}
\begin{center}
\small
\begin{tabular}{| c | c |}
\hline\rule{0pt}{4ex}
Sample Method Usage & Return Value \\
\hline\rule{0pt}{5ex}
\makecell[l]{\texttt{\textbf{int}[] a1 = \{ 1, 2, 3, 4, 100 \};} \\ \texttt{\textbf{int} n1 = centeredAverage(a1);}} & \texttt{3}\\
\hline\rule{0pt}{5ex}
\makecell[l]{\texttt{\textbf{int}[] a2 = \{ 1, 1, 5, 5, 10, 8, 7 \};} \\ \texttt{\textbf{int} n2 = centeredAverage(a2);}} & \texttt{5}\\
\hline\rule{0pt}{5ex}
\makecell[l]{\texttt{\textbf{int}[] a3 = \{ -10, -4, -2, -4, -2, 0 \};} \\ \texttt{\textbf{int} n3 = centeredAverage(a3);}} & \texttt{-3}\\
\hline
\end{tabular}
\end{center}

\vspace*{0.5cm}
\noindent\underline{\textbf{Problem \#2}}
\begin{itemize}
	\item Write a program that has the class name \texttt{Problem2} and that has the \texttt{main} method. Prompt the user to enter an integer \texttt{n}.
	\item Write a method named \texttt{nMinusString} that takes one parameter, an integer \texttt{n} and returns a String array.
	\item The method should create a new String array size \texttt {n}. The method should also populate the array from 0 to n -1.
	\item Create a \texttt{printArray} method that takes a String array and prints out the each elemnt of the array.
	\item Several sample usages are provided for you below. Use the sample usages in the \texttt{main} method to test your code (and use the \texttt{printArray} method to print out the results of calling the \texttt{tenInARow} method!).
\end{itemize}
\begin{center}
\begin{minipage}{3cm}
\begin{lstlisting}[escapechar=@]
Enter an int: @\textbf{7}@
0 1 2 3 4 5 6
\end{lstlisting}
\end{minipage}
\hspace*{.5cm}
\begin{minipage}{3cm}
\begin{lstlisting}[escapechar=@]
Enter an int: @\textbf{9}@
0 1 2 3 4 5 6 7 8
\end{lstlisting}
\end{minipage}
\hspace*{.5cm}
\begin{minipage}{3cm}
\begin{lstlisting}[escapechar=@]
Enter an int: @\textbf{5}@
0 1 2 3 4 
\end{lstlisting}

\end{minipage}\\


\end{center}

\end{document}