\documentclass[12pt]{article}
\usepackage[margin=0.8in]{geometry}
\usepackage{amsmath}
\usepackage{enumerate}
\usepackage{enumitem}
\usepackage[pdftex]{graphicx}
\usepackage{hyperref}
\usepackage{framed}
\usepackage{upquote}
\usepackage{listings}
\usepackage{makecell}
\usepackage{pxfonts}
\usepackage[T1]{fontenc}
\pagestyle{plain}
\usepackage{booktabs}

\lstset{%
  language=Java,
  basicstyle=\fontsize{9}{11}\selectfont\ttfamily,
  keywordstyle=\bfseries,
  breaklines=true,
  commentstyle = \fontsize{9}{11}\selectfont\ttfamily,
  columns=fullflexible,
  frame=single,
  showstringspaces=false,
  tabsize=4,
  morekeywords = {main},
}

\begin{document}
\begin{center}
	\textbf{CS-200: Programming I}\\
	\textbf{Spring 2017}\\
	\textbf{Northeastern Illinois University}\\
	\textbf{PLTL: Week of 04/17/17}\\
	\textbf{Review}
\end{center}


\noindent\underline{\textbf{Problem \#1}}
\begin{itemize}
	\item Write a program that has the class name \texttt{Problem1} and that has the \texttt{main} method. Leave the \texttt{main} method empty for now.
	\item Write a method named \texttt{diagSquare} that takes a 2D integer array \texttt{a}, and returns a boolean.
	\item The method should check if both diagonal sums of the square array are the same. If they are, return true.
	\item The program should then print out the boolean value that is returned.
	\item Several sample usages are provided for you below. Use the sample usages in the \texttt{main} method to test your code.
\end{itemize}

\begin{table}[htbp]
  \centering
    \begin{tabular}{|p{26.66em}|c|}
    \toprule
    \multicolumn{1}{|c|}{Sample Method Usage} & Return Value \\
    \midrule
    \texttt{\textbf{int}[][] a1 = \{\{1, 2, 3,  4\},\newline \hspace*{2.85cm} \{5, 6, 7, 8\},\newline \hspace*{2.85cm} \{2, 1, 5, 9\}, \newline \hspace*{2.85cm} \{10, 2, 3, 10\}\}; }& \texttt{\textbf{true}} \\
    \midrule
    \texttt{\textbf{int}[][] a1 = \{\{3, 4, 7,  10\},\newline \hspace*{2.85cm} \{19, 50, 4, 6\},\newline \hspace*{2.85cm} \{2, 5, 11, 74\},\newline \hspace*{2.85cm} \{1, 2, 3, 4\}\}; } & \texttt{\textbf{false}} \\
    \bottomrule
    \end{tabular}%
  \label{tab:addlabel}%
\end{table}%

\vspace*{0.5cm}
\noindent\underline{\textbf{Problem \#2}}
\begin{itemize}
	\item Write a method named \texttt{change2D} that takes a 1D aray of integers and returns a 2D array.
	
	\item The method should create a new 2D array that always has two rows. The method should place the elements from the 1D array that are not divisble by 2 into equally divided number of columns in the 2D array. You may assume that the number of elements taken from the 1D array will always be even.
	\item Create a print array method to display the output of the new 2D array.
	\item Several sample usages are provided for you below. Use the sample usages in the \texttt{main} method to test your code. 
\end{itemize}

\begin{table}[htbp]

  \centering
    \begin{tabular}{|p{28.66em}|p{11em}|}
    \toprule
    \multicolumn{1}{|c|}{Sample Method Usage} & \multicolumn{1}{c|}{Return Value} \\
    \midrule
   \texttt{ \textbf{int}[] a1 = \{2, 5, 37, 103, 94, 71, 67, 99, 43, 21\}; }& \texttt{\{\{5, 37, 103, 71\},\newline \hspace*{.005em} \{67, 99, 43, 21\}\}} \\
    \midrule
   \hspace*{6pt}\texttt{\textbf{int}[] b1 = \{3, 2, 6, 7\}; } & \{\{3\},\newline \hspace*{3pt}\{7\}\} \\
    \bottomrule
    \end{tabular}%
\end{table}%
\end{document}