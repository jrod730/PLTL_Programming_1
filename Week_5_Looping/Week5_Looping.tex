\documentclass[12pt]{article}
\usepackage[margin=0.8in]{geometry}
\usepackage{amsmath}
\usepackage{enumerate}
\usepackage{enumitem}
\usepackage[pdftex]{graphicx}
\usepackage{hyperref}
\usepackage{framed}
\usepackage{listings}
\usepackage{makecell}
\usepackage{pxfonts}
\pagestyle{plain}

\lstset{%
  language=Java,
  basicstyle=\fontsize{9}{11}\selectfont\ttfamily,
  keywordstyle=\bfseries,
  breaklines=true,
  commentstyle = \fontsize{9}{11}\selectfont\ttfamily,
  columns=fullflexible,
  frame=single,
  showstringspaces=false,
  tabsize=4,
  morekeywords = {main},
}

\begin{document}
\begin{center}
	\textbf{CS-200: Programming I}\\
	\textbf{Fall 2017}\\
	\textbf{Northeastern Illinois University}\\
	\textbf{PLTL: Week of 10/03/17}\\
	\textbf{Looping}
\end{center}


\noindent\textbf{Practice Problem \#1}
\begin{itemize}
	\item Write a program that has the class name \texttt{Problem1} and that has the \texttt{main} method. The \texttt{main} method should take user input for one integer.
	\item The create another method called \texttt{isPrime} that is void and takes one integer as a parameter. .
	\item Write a third method named \texttt{check} that returns a boolean and takes an integer as a parameter.
	\item The program should print out \texttt{true} if the number entered by the user is prime and \texttt{false} if otherwise.
	\item Several sample runs are provided for you below. Your output must be formatted \textbf{exactly} like the sample runs below. Note that while your output must be formatted as below, you will not get the same results as this uses random numbers.
\end{itemize}
\begin{center}
\begin{minipage}{8cm}
\begin{lstlisting}[escapechar=@]
Enter n: @\textbf{5}@
true
\end{lstlisting}
\end{minipage}\\
\vspace*{0.5cm}
\begin{minipage}{8cm}
\begin{lstlisting}[escapechar=@]
Enter n: @\textbf{9}@
false
\end{lstlisting}
\end{minipage}
\end{center}

\vspace*{0.5cm}
\noindent\textbf{Practice Problem \#2}
\begin{itemize}
	\item Write a program that has the class name \texttt{Problem2} and that has the \texttt{main} method.
	\item IAsk the user to enter integers until 3 consecutive inreasing integers are enter.
	\item The program should display the sum of all the integers entered.
	\item Several sample runs are provided for you below. Your output must be formatted \textbf{exactly} like the sample runs below.
\end{itemize}
\begin{center}
\begin{minipage}{3cm}
\begin{lstlisting}[escapechar=@]
Enter an int: @\textbf{3}@
Enter an int: @\textbf{4}@
Enter an int: @\textbf{5}@
12
\end{lstlisting}
\end{minipage}
\hspace*{.5cm}
\begin{minipage}{3cm}
\begin{lstlisting}[escapechar=@]
Enter an int: @\textbf{20}@
Enter an int: @\textbf{-2}@
Enter an int: @\textbf{-1}@
Enter an int: @\textbf{0}@
17
\end{lstlisting}

\end{minipage}
\hspace*{.5cm}
\begin{minipage}{3cm}
\begin{lstlisting}[escapechar=@]
Enter an int: @\textbf{1}@
Enter an int: @\textbf{7}@
Enter an int: @\textbf{8}@
Enter an int: @\textbf{2}@
Enter an int: @\textbf{9}@
Enter an int: @\textbf{10}@
Enter an int: @\textbf{11}@
48
\end{lstlisting}

\end{minipage}\\


\end{center}

\vspace*{0.5cm}


\end{document}